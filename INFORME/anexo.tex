\section{Anexos}

\subsection{Tamizado de Agregados iniciales}

%---------------------- TABLA: GRAVA 1 ----------------------
\begin{table}[H]
\centering
\caption{Análisis granulométrico – Grava 1}
\label{tab:grava1}
\small
\begin{tabular}{|c|c|c|c|c|}
\hline
\textbf{Apertura (mm)} & \textbf{ASTM} & \textbf{Retenido [g]} & \textbf{Retenido [\%]} & \textbf{Pasa [\%]} \\ \hline
75   & 3"      & 0   & 0\%  & 100\% \\ \hline
63   & 2 1/2"  & 0   & 0\%  & 100\% \\ \hline
50   & 2"      & 0   & 0\%  & 100\% \\ \hline
37,5 & 1 1/2"  & 60  & 6\%  & 94\%  \\ \hline
25   & 1"      & 320 & 32\% & 62\%  \\ \hline
19   & 3/4"    & 480 & 48\% & 14\%  \\ \hline
12,5 & 1/2"    & 60  & 6\%  & 8\%   \\ \hline
9,5  & 3/8"    & 80  & 8\%  & 0\%   \\ \hline
6,3  & 1/4"    & 0   & 0\%  & 0\%   \\ \hline
4,75 & N°4     & 0   & 0\%  & 0\%   \\ \hline
2,36 & N°8     & 0   & 0\%  & 0\%   \\ \hline
1,18 & N°16    & 0   & 0\%  & 0\%   \\ \hline
0,6  & N°30    & 0   & 0\%  & 0\%   \\ \hline
0,3  & N°50    & 0   & 0\%  & 0\%   \\ \hline
0,15 & N°100   & 0   & 0\%  & 0\%   \\ \hline
\textbf{Residuo} &     & 0   & -    & -     \\ \hline
\textbf{TOTAL}  &     & 1000& 100\%& -     \\ \hline
\end{tabular}
\end{table}

%---------------------- TABLA: GRAVILLA 2 ----------------------
\begin{table}[H]
\centering
\caption{Análisis granulométrico – Gravilla 2}
\label{tab:gravilla2}
\small
\begin{tabular}{|c|c|c|c|c|}
\hline
\textbf{Apertura (mm)} & \textbf{ASTM} & \textbf{Retenido [g]} & \textbf{Retenido [\%]} & \textbf{Pasa [\%]} \\ \hline
75   & 3"      & 0   & 0\%  & 100\% \\ \hline
63   & 2 1/2"  & 0   & 0\%  & 100\% \\ \hline
50   & 2"      & 0   & 0\%  & 100\% \\ \hline
37,5 & 1 1/2"  & 0   & 0\%  & 100\% \\ \hline
25   & 1"      & 0   & 0\%  & 100\% \\ \hline
19   & 3/4"    & 50  & 5\%  & 95\%  \\ \hline
12,5 & 1/2"    & 350 & 35\% & 60\%  \\ \hline
9,5  & 3/8"    & 120 & 12\% & 48\%  \\ \hline
6,3  & 1/4"    & 280 & 28\% & 20\%  \\ \hline
4,75 & N°4     & 80  & 8\%  & 12\%  \\ \hline
2,36 & N°8     & 120 & 12\% & 0\%   \\ \hline
1,18 & N°16    & 0   & 0\%  & 0\%   \\ \hline
0,6  & N°30    & 0   & 0\%  & 0\%   \\ \hline
0,3  & N°50    & 0   & 0\%  & 0\%   \\ \hline
0,15 & N°100   & 0   & 0\%  & 0\%   \\ \hline
\textbf{Residuo} &     & 0   & -    & -     \\ \hline
\textbf{TOTAL}  &     & 1000& 100\%& -     \\ \hline
\end{tabular}
\end{table}

%---------------------- TABLA: ARENA 3 ----------------------
\begin{table}[H]
\centering
\caption{Análisis granulométrico – Arena 3}
\label{tab:arena3}
\small
\begin{tabular}{|c|c|c|c|c|}
\hline
\textbf{Apertura (mm)} & \textbf{ASTM} & \textbf{Retenido [g]} & \textbf{Retenido [\%]} & \textbf{Pasa [\%]} \\ \hline
75   & 3"      & 0   & 0\%  & 100\% \\ \hline
63   & 2 1/2"  & 0   & 0\%  & 100\% \\ \hline
50   & 2"      & 0   & 0\%  & 100\% \\ \hline
37,5 & 1 1/2"  & 0   & 0\%  & 100\% \\ \hline
25   & 1"      & 0   & 0\%  & 100\% \\ \hline
19   & 3/4"    & 0   & 0\%  & 100\% \\ \hline
12,5 & 1/2"    & 0   & 0\%  & 100\% \\ \hline
9,5  & 3/8"    & 0   & 0\%  & 100\% \\ \hline
6,3  & 1/4"    & 40  & 4\%  & 96\%  \\ \hline
4,75 & N°4     & 280 & 28\% & 68\%  \\ \hline
2,36 & N°8     & 130 & 13\% & 55\%  \\ \hline
1,18 & N°16    & 170 & 17\% & 38\%  \\ \hline
0,6  & N°30    & 110 & 11\% & 27\%  \\ \hline
0,3  & N°50    & 170 & 17\% & 10\%  \\ \hline
0,15 & N°100   & 60  & 6\%  & 4\%   \\ \hline
\textbf{Residuo} &     & 40  & -    & -     \\ \hline
\textbf{TOTAL}  &     & 1000& 100\%& -     \\ \hline
\end{tabular}
\end{table}

\subsection{Tablas de especificaciones ACI 211.1 y ACI 318}

\begin{table}[H]
\centering
\caption{Rangos típicos de asentamiento sin aditivos reductores de agua.}
\label{tab:rangos-asentamiento}
\setlength{\tabcolsep}{6pt}
\renewcommand{\arraystretch}{1.15}
\small
\makebox[\textwidth][c]{%
\begin{tabular}{@{} c l @{}} % dos columnas
\toprule
\textbf{Rango de slump (in)} & \textbf{Tipo de construcción} \\
\midrule
1--4 & Slipformed \\
2--4 & Mass concrete \\
2--5 & Pavimentos, losas, fundaciones simples y armadas, muros de subestructura \\
3--5 & Vigas, muros armados y columnas \\
\bottomrule
\end{tabular}%
}
\\Fuente: ACI 211.1-22.
\end{table}

\begin{table}[H]
\centering
\caption{Contenido aproximado de agua de amasado y aire según TMA y \textit{slump}.}
\label{tab:contenido-agua-aire}
\setlength{\tabcolsep}{6pt}
\renewcommand{\arraystretch}{1.15}
\small
\makebox[\textwidth][c]{%
\begin{tabular}{@{} c c c c c c c c @{}} % ocho columnas
\toprule
\textbf{Slump (in)} & \textbf{3/8''} & \textbf{1/2''} & \textbf{3/4''} & \textbf{1''} & \textbf{1-1/2''} & \textbf{2''} & \textbf{3''} \\
\midrule
\multicolumn{8}{c}{\textbf{Hormigón sin aire incorporado (lb/yd$^3$)}} \\
\midrule
1--2 & 350 & 335 & 315 & 305 & 275 & 260 & 220 \\
3--4 & 385 & 365 & 340 & 325 & 300 & 285 & 250 \\
5--6 & 410 & 375 & 350 & 340 & 320 & 295 & 255 \\
6--7 & 430 & 385 & 360 & 340 & 330 & 300 & 270 \\
\midrule
\multicolumn{8}{c}{\textbf{Hormigón con aire incorporado (lb/yd$^3$)}} \\
\midrule
1--2 & 305 & 295 & 280 & 270 & 240 & 230 & 205 \\
3--4 & 340 & 325 & 305 & 295 & 275 & 265 & 225 \\
5--6 & 355 & 335 & 315 & 300 & 280 & 270 & 240 \\
6--7 & 365 & 340 & 315 & 310 & 280 & 280 & 250 \\
\bottomrule
\end{tabular}%
}
\\Aire atrapado (no aireado): 3\%, 2.5\%, 2\%, 1.5\%, 1\%, 0.5\%, 0.3\% (según TMA). \\
Requerido aire total: F1 = 6\%, F2/F3 = 7.5\%. \\
Fuente: ACI 211.1-22.
\end{table}

\begin{table}[H]
\centering
\caption{Valores de $t$ según probabilidad de defectuosos.}
\label{tab:valores-t}
\setlength{\tabcolsep}{6pt}
\renewcommand{\arraystretch}{1.15}
\small
\makebox[\textwidth][c]{%
\begin{tabular}{@{} c c @{}} 
\toprule
\textbf{Probabilidad de defectuosos (\%)} & \textbf{$t$} \\
\midrule
1   & 2.33 \\
2.5 & 1.96 \\
5   & \textbf{1.645} \\
10  & 1.282 \\
20  & 0.842 \\
\bottomrule
\end{tabular}%
}
\\Fuente: ACI 318.
\end{table}

\begin{table}[H]
\centering
\caption{Condiciones de obra y desviación estándar $s$ (MPa).}
\label{tab:condiciones-obra}
\setlength{\tabcolsep}{6pt}
\renewcommand{\arraystretch}{1.15}
\small
\makebox[\textwidth][c]{%
\begin{tabular}{@{} c c c l @{}} 
\toprule
\textbf{Condiciones} & \textbf{$s \leq H15$} & \textbf{$s > H15$} & \textbf{Definición} \\
\midrule
Regulares   & 8.0 & --- & Control deficiente, solo grado $\leq$ H15 \\
Medias      & 6.0 & 7.0 & Dosificación de volumen controlado; control esporádico \\
Buenas      & 4.0 & 5.0 & Dosificación en peso o volumen controlado; control permanente \\
Muy buenas  & 3.0 & 4.0 & Dosificación en peso; laboratorio en faena; control permanente \\
\bottomrule
\end{tabular}%
}
\\Fuente: ACI 318.
\end{table}

\begin{table}[H]
\centering
\caption{Rangos recomendados de asentamiento según tipo de construcción (mm).}
\label{tab:rangos-asentamiento-mm}
\setlength{\tabcolsep}{6pt}
\renewcommand{\arraystretch}{1.15}
\small
\makebox[\textwidth][c]{%
\begin{tabular}{@{} l c c @{}} 
\toprule
\textbf{Tipo de construcción} & \textbf{Slump máx. (mm)} & \textbf{Slump mín. (mm)} \\
\midrule
Fundaciones y muros armados                    & 75  & 25 \\
Fundaciones simples, cajones, muros subterráneos & 75  & 25 \\
Vigas y muros armados                          & 100 & 25 \\
Columnas                                       & 100 & 25 \\
Pavimentos y losas                             & 75  & 25 \\
Hormigón masivo                                & 75  & 25 \\
\bottomrule
\end{tabular}%
}
\\Fuente: ACI 211.1-22.
\end{table}

\begin{table}[H]
\centering
\caption{Requisitos por clase de exposición.}
\label{tab:clase-exposicion}
\setlength{\tabcolsep}{6pt}
\renewcommand{\arraystretch}{1.15}
\small
\makebox[\textwidth][c]{%
\begin{tabular}{@{} c c c l c @{}} 
\toprule
\textbf{Clase} & \textbf{W/C máx.} & \textbf{$f'_c$ mín. (MPa)} & \textbf{Requisitos adicionales} & \textbf{Límite de cloruros} \\
\midrule
F0 & NA   & 17 & Ninguno                  & --- \\
F1 & 0.45 & 31 & Ver tabla específica      & --- \\
F2 & 0.45 & 31 & Ver tabla específica      & --- \\
F3 & 0.45 & 31 & Ver tabla específica + SCM & --- \\
S0 & NA   & 17 & Ninguno                  & --- \\
S1 & 0.50 & 28 & Cemento II (MS)          & --- \\
S2 & 0.45 & 31 & Cemento V (HS)           & --- \\
S3 & 0.45 & 31 & V + puzolana o escoria   & --- \\
C0 & NA   & 17 & ---                      & 1.0 (HA) / 0.06 (HP) \\
C1 & NA   & 17 & ---                      & 0.3 (HA) / 0.06 (HP) \\
C2 & 0.40 & 35 & ---                      & 0.15 (HA) / 0.06 (HP) \\
\bottomrule
\end{tabular}%
}
\\Fuente: ACI 318.
\end{table}

\begin{table}[H]
\centering
\caption{Relación tipo de estructura – condiciones de exposición y relación W/C.}
\label{tab:estructura-wc}
\setlength{\tabcolsep}{6pt}
\renewcommand{\arraystretch}{1.15}
\small
\makebox[\textwidth][c]{%
\begin{tabular}{@{} l c c c c @{}} 
\toprule
\textbf{Tipo de estructura} & \textbf{Hielo-deshielo al aire} & \textbf{Bajo agua dulce} & \textbf{Bajo agua con sulfatos} & \textbf{Clima suave al aire} \\
\midrule
Secciones delgadas, poco recubrimiento      & 0.50   & 0.45   & 0.40   & 0.55 \\
Secciones moderadas                        & 0.55   & 0.50   & 0.45   & 0.55 \\
Exterior de hormigón simple                 & 0.60   & 0.50   & 0.45   & 0.55 \\
Hormigonado bajo agua                       & ---    & 0.45   & 0.45   & ---  \\
Losas sobre el suelo                        & 0.55   & ---    & ---    & ---  \\
Hormigón protegido a la intemperie          & ver (2)& ---    & ---    & ver (2) \\
Protegido pero expuesto a hielo-deshielo antes & 0.55 & ---    & ---    & ver (2) \\
\bottomrule
\end{tabular}%
}
\\Fuente: ACI 318.
\end{table}






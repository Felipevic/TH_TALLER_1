\section{Introducción}

La dosificación de hormigón es un proceso fundamental en la construcción, ya que determina las proporciones de los materiales que componen la mezcla, como los áridos, el cemento y el agua. 
Un diseño adecuado de la mezcla garantiza la trabajabilidad e influye directamente en la resistencia, durabilidad y desempeño del hormigón en diversas condiciones de exposición. 
En este informe se abordarán los métodos de dosificación de hormigón según las normas NCh 163 y ACI 211.1S-91, particularmente en la optimización granulométrica, la cual busca mejorar la distribución del tamaño de las partículas de los áridos para lograr una mezcla más eficiente y funcional.

Las normas mencionadas anteriormente sirvieron para dar contexto y realizar el desarrollo del taller. El método ACI se utilizó para la dosificación inicial, siguiendo especificaciones detalladas. Posteriormente, se realizaron correcciones y ajustes granulométricos para optimizar dicha mezcla, utilizando métodos como la curva de Fuller-Thompson y la curva Tarántula, garantizando una mezcla homogénea y de alto rendimiento.

\subsection{Objetivos}

\begin{enumerate}
    \item Optimizar y corregir la dosificación de un hormigón.
    \item Aplicar diferentes métodos de optimización granulométrica como TMA y Fuller-Thompson.
    \item Corroborar la trabajabilidad de una mezcla comparandola con la curva Tarántula.
\end{enumerate}
\section{Conclusión}

En este taller, se aplicó el método de la Curva Tarántula para optimizar la distribución granulométrica de los áridos en la mezcla de hormigón. Tras realizar el análisis granulométrico y calcular los porcentajes pasantes, se construyeron las curvas de TMA y Fuller-Thompson, que se compararon con los límites superior e inferior establecidos por la Curva Tarántula. Ambos métodos demostraron ser técnicamente válidos, con resultados que se mantuvieron dentro del rango acotado, lo que asegura una mezcla cohesiva, estable y con buena trabajabilidad.

El análisis reológico y la verificación de la mezcla confirmaron que las distribuciones granulométricas obtenidas cumplen con los requisitos de estabilidad y bombeabilidad. Además, se destacó la importancia de ajustar las proporciones de los áridos en función de los resultados obtenidos, garantizando la idoneidad de la mezcla para su uso en obra. En resumen, la aplicación de la Curva Tarántula permitió asegurar una mezcla de hormigón adecuada, optimizando el uso de los áridos disponibles y mejorando las características de la mezcla en términos de resistencia y durabilidad.
